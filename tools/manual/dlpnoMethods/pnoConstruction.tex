\section{Pseudo-natural Orbital Construction}
The purpose of this section is to explain the pair natural orbital (PNO), triple-natural orbital (TNO) and
quasi canonical projected atomic orbital (QC-PAO) construction. We will refer to PNOs and TNOs as pseudo-natural
orbitals in the following section. We will always assume restricted Hartree-Fock orbitals as our reference orbitals.
This section is based on the following references~\cite{Pulay1983,Neese2009b,Riplinger2013a}.

\subsection{Redundant Projected-Atomic Orbital Construction}
In every local correlation method, some sort of local representation of the virtual orbitals is necessary.
Since \textsc{Serenity} mainly implements Domain-based Local Pair Natural Orbital (DLPNO)-based correlation
methods, the virtual space is constructed from projected atomic orbitals (PAOs).
These PAOs $\{|\tilde{\mu}\rangle\}$ are given by
\begin{align}
  |\tilde{\mu}\rangle = \left(1-\sum_{i} |i\rangle\langle i|\right)|\mu\rangle,
  \label{eq:pao}
\end{align}
where the sum runs over all occupied orbitals $|i\rangle$.

These PAOs represent a virtual orbital set that fully spans the virtual orbital space, is linear dependent
and the functions are not necessarily normalized.
The coefficients $\pmb{C}^{\mathrm{PAO}}$ of the redundant, unnormalized PAOs are given in matrix
notation by
\begin{align}
  \pmb{C}^{\mathrm{PAO}} = \pmb{1}-\frac{1}{2} \pmb{D}\pmb{S}^\mathrm{AO},
\end{align}
where $\pmb{1}$ is the identity matrix, $\pmb{D}$ the density matrix in atomic orbital (AO) basis and
$\pmb{S}^\mathrm{AO}$ the overlap matrix in AO basis.

These PAOs are refined further by omitting functions which have a negligible norm
[e.g. $1$s basis functions may be fully removed by the projection in Eq.~(\ref{eq:pao})]
and normalizing all remaining functions.
We will refer to the resulting PAOs as redundant PAOs, since the linear dependencies within the function
sets are still present.

\subsection{Redundant Projected-Atomic Orbital Selection}
In order to motivate the truncation of the PAO space for a specific orbital, orbital pair or orbital triple,
we consider the M{\o}ller--Plesset perturbation theory (MP2) pair energy $\epsilon_{ij}^\mathrm{MP2}$ for
a given pair $ij$. This energy is given by~\cite{Pulay1983}
\begin{align}
  \epsilon_{ij}^\mathrm{MP2} = \frac{1}{1+\delta_{ij}}\sum_{ab}{(ia|jb)}(4t^{ij}_{ab}-2t^{ij}_{ba}),
  \label{eq:MP2PairEnergy}
\end{align}
where the two-electron-exchange integral $(ia|jb)$ is given in $(11|22)$ notation, referring to the integration
variables. The $t^{ij}_{ab}$ are the MP2 amplitudes. The virtual functions are denoted by $a$ and $b$.
The only restriction for the choice in virtual functions is that they are orthogonal to the occupied orbital set.
It is obvious from Eq.~(\ref{eq:MP2PairEnergy}) that terms in the sum can only contribute, if $i$ and $a$ as well as $j$
and $b$ have non-zero differential overlap. Thus, PAOs are assigned to occupied orbitals $i\rightarrow [i]$.
We call $[i]$ the PAO domain of $i$. The PAO domains of an orbital pair $[ij]$ are constructed
as the union of the PAO domains of the individual orbitals $[ij]=[i]\cup [j]$. The same is done for orbital
triples ($[ijk]=[i]\cup [j]\cup [k]$).

\subsection{Quasi-Canonical Projected-Atomic Orbital Construction}
Within the truncated, redundant PAO sets, the PAOs are orthogonalized by constructing the PAO--PAO overlap matrix
in the PAO domain and diagonalizing it. The resulting eigenvectors of the PAO--PAO overlap matrix are
truncated by eigenvalue in order to remove (nearly) linear dependent vectors.

Then, the Fock matrix is transformed into the basis of these non-redundant PAOs and diagonalized. The resulting functions
are now called quasi-canonical PAOs (QC-PAOs). This quasi-canonicalization is extremely convenient for local
second order M{\o}ller--Plesset perturbation theory (MP2), since the iterative amplitude optimization
converge exceptionally well. Furthermore, the QC-PAOs are needed to calculate semi-canonical MP2 (SC-MP2)
amplitudes, which will become important in the next step.

\subsection{Pseudo Natural Orbital Construction}

The QC-PAO sets which are necessary to get reasonable, accurate results are often very large. Thus, the size of the
virtual orbital space is reduced further by constructing pseudo natural orbitals. For orbital pairs, we can easily
motivate this by reconsidering Eq.~(\ref{eq:MP2PairEnergy}), which we already used in order to reduce the size of
the initial redundant PAO sets. The idea is to find a reasonable approximation to the MP2 amplitudes $\pmb{t}_{ij}$
and then use this information to find a basis in which the sum in Eq.~(\ref{eq:MP2PairEnergy}) becomes very
compact~\cite{Neese2009b}.

As an approximation to the MP2 amplitudes SC-MP2 amplitudes are used. The latter are given by
\begin{align}
  t^{ij}_{\tilde{\mu},\tilde{\nu}} &= -\frac{(i\tilde{\mu}|j\tilde{\nu})}{{\epsilon_{\tilde{\mu}}}+{\epsilon_{\tilde{\nu}}}-F_{ii}-F_{jj}},
\end{align}
where $\epsilon_{\tilde{\mu}}$ and $\epsilon_{\tilde{\nu}}$ are the eigenvalues of the QC-PAOs and $F_{ii}$ and $F_{jj}$
are diagonal Fock matrix elements.
From these amplitudes, natural orbitals are constructed by diagonalizing the pair density matrix $\pmb{D}^{ij}$ given
by
\begin{align}
  \mathbf{D}^{ij} &= \tilde{\mathbf{t}}^{ij}\mathbf{t}^{ij\dagger} + \tilde{\mathbf{t}}^{ij\dagger}\mathbf{t}^{ij},
\end{align}
where $\tilde{t}^{ij}_{\tilde{\mu},\tilde{\nu}}$ is given by
\begin{align}
  \tilde{t}^{ij}_{\tilde{\mu},\tilde{\nu}}  &= \frac{1}{1+\delta_{ij}}(4t^{ij}_{\tilde{\mu}\tilde{\nu}}-2t^{ij}_{\tilde{\nu}\tilde{\mu}}).
\end{align}

The eigenvectors of $\mathbf{D}^{ij}$ are truncated by their eigenvalues and the remaining functions are
rotated such that they diagonalize the Fock matrix.

The pseudo-natural orbitals for singles of orbital $i$ are constructed from the PNOs of the
diagonal pair $ii$ by adjusting the truncation threshold for the PNOs.

For orbital triples, a triple-density matrix $\mathbf{D}^{ijk}$ is constructed from the pair density matrices
of the orbital pairs~\cite{Riplinger2013a} $ij$, $ik$ and $jk$ as
\begin{align}
  \mathbf{D}^{ijk} = \frac{1}{3} \left( \mathbf{D}^{ij} + \mathbf{D}^{ik} +\mathbf{D}^{jk} \right).
\end{align}
The remaining procedure for the Triple-natural Orbital (TNO) construction is the same as for the PNOs.
However, in case of a triples correction to the correlation energy, the converged amplitudes of the orbital
pairs are usually used for the TNO construction instead of the SC-MP2 guess.
